\documentclass[]{article}
\usepackage{tikz,lipsum,lmodern}
\usepackage[most]{tcolorbox}
\usepackage[top=0pt,bottom=1cm,left=2cm,right=2cm,heightrounded]{geometry}

%opening
\title{Interactive Session Jupiter to launch openfold environment gui example}
\author{Jack Raymakers}
\begin{document}
\maketitle
\vspace{-20pt} 
\setlength\parindent{0pt}
%\section{}
Jupiter\\
\\
This app will launch a Jupyter server using Python.\\\\
\emph{Jupyter interface}
\begin{tcolorbox}[colback=white,colframe=gray!75!black]{
		Lab
}
\end{tcolorbox}
This defines the interface of Jupyter you want to start (Notebook or Lab).\\\\
\emph{Jupyter Python}
\begin{tcolorbox}[colback=white,colframe=gray!75!black]{
Custom(Specify below)
}
\end{tcolorbox}

This defines the Python distribution of Jupyter you want to start.\\\\

\emph{Custom Python Environment}
\begin{tcolorbox}[colback=white,colframe=gray!75!black]{
\begin{verbatim}
module load anaconda/miniconda3_4.8.3-jupyter
source /common/software/install/manual/miniforge3/24.3/etc/profile.d/conda.sh
conda activate openfold-env
echo "CONDA_PREFIX=$CONDA_PREFIX"
echo "PATH=$PATH"
\end{verbatim}	
	}
\end{tcolorbox}
Enter commands (module load, source activate, etc) to create your desired Jupyter environment; jupyter MUST be on your path and either notebook or jupyterlab installed in your Python environment.\\\\

\emph{Notebook Root}
\begin{tcolorbox}[colback=white,colframe=gray!75!black]{
/scratch.global/rayma017/Workspace/FinalProject/
}
\end{tcolorbox}
The root notebook folder to use. If empty, defaults to your home directory.\\\\
\emph{Account}
\begin{tcolorbox}[colback=white,colframe=gray!75!black]{
ee4940
}
\end{tcolorbox}
Select Slurm account to charge, will also set Jupyter's GID to this group.\\\\
\emph{Resources}
\begin{tcolorbox}[colback=white,colframe=gray!75!black]{
Interactive GPU-16 cores,60 GB, 100 GB local scratch,1 A40
}
\end{tcolorbox}
\emph{Time Limit}
\begin{tcolorbox}[colback=white,colframe=gray!75!black]{
24 hours
}
\end{tcolorbox}
Shorter times will probably start faster
[] I would like to receive an email when the session starts

\newpage
\newgeometry{top=25mm, bottom=25mm} 

For the Gui examples two notebooks were create to run on the MSI:\\
The openfold-env has been already created.\\\\

Run once:\\
conda activate openfold-env\\
python -m ipykernel install --name openfold-env --display-name "Python openfold-env"\\\\

launch a jupyter session with the setup from  prior page.\\
		\begin{itemize}{}
			\item{example\_notebook.ipynb}
			\item{example\_py3dmol.ippynb}
		\end{itemize}
\end{document}
