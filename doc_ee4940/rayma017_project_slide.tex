\documentclass[xcolor = svgnames,hyperref = {colorlinks = true},]{beamer}
\usepackage{tikz,lipsum,lmodern}
\usepackage[most]{tcolorbox}
\usepackage{listofitems} % for \readlist to create arrays
\usetheme{Madrid}
\usecolortheme{beaver}
\usecolortheme{beaver}
\usefonttheme{structurebold}
\usepackage[colorlinks]{hyperref}
\title{Intergation of Openfold into MSI}
\author{Jack Raymakers}
\institute{University of Minnesota}
\date{\today}

%linkcolor [red]
%anchorcolor [black]
%citecolor [green]
%filecolor [cyan]
%menucolor [red]
%runcolor [cyan - same as file color]
%urlcolor [magenta]
%allcolors -- use this if you want to set all links to the same color
\newcommand{\sectframe}[1]{\begin{frame}{#1}
		\section{#1}
}
%\begin{frame}{#1}
%\section{#1}
\newenvironment{VerbExample}
{\semiverbatim}
{\endsemiverbatim}

\begin{document}
\begin{frame}
\titlepage
%\textbf{\hyperlinkslidenext{\beamerbutton{next}}}

\end{frame}

\begin{frame}
\frametitle{Table of Contents}
\tableofcontents
\end{frame}


\sectframe{Objective and Scope}
Create an environment for the  openfold project from the aqlaboratory/openfold repository, implement using the GPUs on MSI for Jupiter infrastructure and execute unit test and create a prediction using the openfold model. Document all steps.
 \end{frame}

\sectframe{Required Environment}
	\begin{tcolorbox}[colback=blue!5!white,colframe=blue!75!black]
		\centering{Create a ticket}\vspace{10pt}
	\setlength{\itemindent}{-.5in}
	\begin{itemize}{}		
		
		\item{Request a CSE Labs account,\\
			\hspace{10pt}if you don’t already have one}
		\item{Request a MSI services}
		\item{Optional Virtual Online Linux Environment (VOLE)}
	\end{itemize}
\vspace{10pt} URL to Create ticket\\
\url
{https://tdx.umn.edu/TDClient/31/Portal/Requests/ServiceDet?ID=237}
	\begin{itemize}{}			
	\item{Install AnyConnect VPN Client Software(per os)}
		\url{https://software.cisco.com/download/home/286330811/type/282364313/release/5.1.8.122?i=!pp}
\end{itemize}
\end{tcolorbox}
\end{frame}	

\sectframe{open terminal session using vpn}
	\begin{tcolorbox}[colback=blue!5!white,colframe=blue!75!black]
	\begin{itemize}{}
	\item{Launch vpn (open AnyConnect-UofMvpnFull,username,passward)}
{\color{red}~\$}/opt/cisco/secureclient/bin/vpn or (vpnui for gui)
{\color{cyan}VPN}tc-vpn-1.vpn.umn.edu\\
{\color{cyan}Group:} [AnyConnect-UofMvpnFull]\\
{\color{cyan}username:}\$USER\\ 
{\color{cyan}Password:}\$secret \\	
	\item{open session to connect to MSI (ssh exmaple) }\\
ssh username@agate.msi.umn.edu\\
{\color{cyan}Password:}
%https://github.com/aqlaboratory/openfold
\end{itemize}
\end{tcolorbox}
\end{frame}


\sectframe{Make Project path and clone repos}
\begin{tcolorbox}[colback=blue!5!white,colframe=blue!75!black]
	\begin{itemize}{}
		\item{Create Project Workspace and change directory}\\
		 example mkdir -p /scratch.global/\$USER/git\\
		 example cd /scratch.global/\$USER/git/
		\item{full path for example}
		\textbf{\textit{fullpath}} is /scratch.global/\$USER/git 
		\item{clone repo's}\\
git clone --recursive https://github.com/NVIDIA/dllogger.git\\	
git clone --recursive https://github.com/Dao-AILab/flash-attention\\
git clone https://github.com/NVIDIA/cutlass --depth 1
		\item{clone openfold from custimize repo}\\
git clone https://github.com/rayma017/openfold
		\item{The start point was: aqlaboratory/openfold -b pl\_updates }	
	\end{itemize}
\end{tcolorbox}
\end{frame}

\sectframe{Use mamba to create openfold-env}
\begin{tcolorbox}[colback=blue!5!white,colframe=blue!75!black]
	\begin{itemize}{}
		\item{customize .condarc to save space on home drive}\\
		conda config --set auto\_activate\_base false\\
		conda config --add envs\_dirs /scratch.global/\$USER/.conda/envs\_dir\\
		conda config --add pkgs\_dirs /scratch.global/\$USER/.conda/pkgs\_dir	
		\item{run mamba --cd openfold}\\
		./environment.yml.sh
		\item{if you have a completion issue navigate to project dir use mamba with update}\\
		mamba env update -f environment.yml
		\item{after successful creation you be provide with activate sequence}\\	
	\end{itemize}
\end{tcolorbox}
\end{frame}

\sectframe{Quick Check}
\begin{tcolorbox}[colback=blue!5!white,colframe=blue!75!black]
	\begin{itemize}{}
		\item{Hop to GPU node}\\
		srun -N 1 -n 1 --mem=32g -t 240 --gres=gpu:a40:1 --tmp=100g -p interactive-gpu --x11 xterm  -fa default -fs 1
		\item{Run... conda deactivate}
		\item{Run... conda activate openfold-env}
		\item{Run... python test\_cuda.py}
	\end{itemize}
\end{tcolorbox}
\end{frame}



\sectframe{copy model parameters to your update openfold/resources}
\begin{tcolorbox}[colback=blue!5!white,colframe=blue!75!black]
	\begin{itemize}{}
		\item{Navigate to inside repo and execute}
		\item{update conda environmet}
		source scripts/update\_conda\_vars
		\item{scripts/download\_openfold\_params.sh resources/}	
		\item{scripts/download\_openfold\_solseq\_params.sh resources/}
		\item{scripts/download\_alphafold \_params.sh resources/}
	\end{itemize}
\end{tcolorbox}
\end{frame}


\subsection{Run unit test}
\begin{frame}[fragile]
	\begin{tcolorbox}[colback=blue!5!white,colframe=blue!75!black]
	\begin{itemize}{}
			\item{Navigate to inside openfold repo exec:}
			\item{run unit test}\\
srun -N 1 -n 10 --mem=32g -t 100 --gres=gpu:a40:1 $\backslash$
--tmp=100g -p interactive-gpu --pty bash\\
conda actavate openfold-env
scripts/run\_unit\_tests.sh
			\item{Make prediction using example}\\
cd tests/test\_data
\# Please note the main database is not install\\
../../examples/monomer/inference.sh
		\end{itemize}
	\end{tcolorbox}
\end{frame}

\sectframe{Current state and Notes}
	\begin{tcolorbox}[colback=blue!5!white,colframe=blue!75!black]
		\begin{itemize}{}
			\item{openfold model, openfold parameters, alphafold parameters,utilizes and
			tests are installed}
			\item{Several Terabytes of date from protein database are install}
			\item{Several Terabytes of training data are install}\\
			\item{ran unit test}
			\item{ran prediction with model}
			\item{updated original environment to run on MSI}
			\item{created a git with updates}
			\item{TODO}
			\item{Create Jupiter note example on MSI}
			\item{Verify git/raym017/openfold with fresh install}
		\end{itemize}
	\end{tcolorbox}
\end{frame}

\sectframe{Notes and Questions}
\begin{tcolorbox}[colback=blue!5!white,colframe=blue!75!black]
	\begin{itemize}{}
		\item{Resolving conflict cuda ,pytorch and requirement from other install}
		\item{Downloading data exceeded time allocation}
		\item{Urls that no longer existed}
		\item{GUI: launching plots html,raster}
		\item{Resolving help gain structure of the database}
		\item{Questions:}
	\end{itemize}
\end{tcolorbox}
\end{frame}
\end{document}
%srun -N 1 -n 10 --mem=32g -t 100 --gres=gpu:a40:1 --tmp=100g -p interactive-gpu --pty bash


%273  apptainer pull docker:docker.io/nvidia/cuda:10.2-cudnn8-runtime-ubuntu18.04
%274  apptainer pull docker://docker.io/nvidia/cuda:10.2-cudnn8-runtime-ubuntu18.04
%275  apptainer pull docker://nvidia/cuda:10.2-cudnn8-runtime-ubuntu18.04

%  286  apptainer build runtime_latest.sif library://runtime_latest

% apptainer shell  -f --overlay my_overlay --dns 8.8.8.8 pytorch_24.04-py3.sif

%apptainer sif info pytorch_24.04-py3.sif 
%apptainer run --dns 8.8.8.8 --overlay my_overlay git/runtime_latest.sif

% rayma017@ahl03 [/scratch.global/rayma017/git] % curl -L -O "https://github.com/conda-forge/miniforge/releases/latest/download/Miniforge3-$(uname)-$(uname -m).sh"

% bash  Miniforge3-$(uname)-$(uname -m).sh

% Do you wish to update your shell profile to automatically initialize conda?
%This will activate conda on startup and change the command prompt when activated.
%If you'd prefer that conda's base environment not be activated on startup,
%run the following command when conda is activated:

%

%You can undo this by running `conda init --reverse $SHELL`? [yes|no]
%\subsection{Create first cut of requirements}
%\begin{frame}[fragile]
%		\begin{tcolorbox}[colback=blue!5!white,colframe=blue!75!black]
	%			\begin{itemize}{}
		%				\item{hop to node with gpu slum job}\\
		%
		%				srun -N 1 -n 10 --mem=32g -t 100 --gres=gpu:a40:1 $\backslash$  \\
		%				--tmp=100g -p interactive-gpu --pty bash\\
		%				\item{change directory to the openfold project directory}\\
		%				cd /scratch.global/\$USER/git/openfold
		%				\item{Determine the dependencies and determine what is installed}
		%\begin{verbatim}
		%INSTALLEDLIST="$(
		%pip3 list | sed  's/  *.*//' | \
		%tail -n +3 piplist.txt | \
		%tr '\n' '|' | \
		%sed -E 's/.$//')"
		%tail -n +7 environment.yml | \
		%sed -E "/$INSTALLEDLIST/d" | \
		%sed -E 's/  *\- //' | \
		%sed -E 's/bioconda:://' | \
		%sed -E 's/=.*$//'> pipinstall.txt
		%tail -n +3 pipinstall.txt
		%\end{verbatim}					%
		%			\end{itemize}
	
	%		\end{tcolorbox}
%\end{frame}
mkdir run_dir
cp -r opedfold/scripts run_dir
 
535  scripts/run_unit_tests.sh
#!/usr/bin/env bash
sed -i '2 conda shell.bash activate openfold-env > /dev/null' inference.sh
sed -i '$a conda shell.bash deactivate > /dev/null' inference.sh
set -e
conda activate openfold-env
## DO SOME STUFF ##
deactivate

conda install -n base conda-libmamba-solver
conda config --set solver libmamba
conda install -n openfold-env nb_conda --solver classic